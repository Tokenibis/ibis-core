\documentclass{article}
\usepackage{parskip}
\usepackage{titlesec}
\usepackage{xcolor}

\title{\vspace{-5ex}Ibis Code Overview}
\date{\vspace{-8ex}\today}

\titleformat{\section}[block]{\bfseries\filcenter}{}{1em}{}

\begin{document}
\maketitle
\noindent\rule{\textwidth}{1pt}

This document serves as a detailed overview of the design goals and decisions
made in implementing the Ibis contracts. For a higher-level explanation of the
purpose of the Ibis platform, please refer to the white paper and blog.

\section{Overview}

The Ibis contracts implement an ERC20 designed to facilitate the functionality
specified in the white paper. In the expected case, these are limited to the
routine operations of token exchange, transfers, and new charity approvals by
the Ibis organization.

The majority of the code is intended to ensure robustness and security. In
designing the system, we are particularly mindful of the following general
threats:

\begin{itemize}
\item Key Theft: We design varying layers of defense-in-depth to mitigate damage
  by potential hacking up to and including total compromise of owner addresses.
\item Economic Abuse: We implement a control mechanism to temporarily freeze
  undesirable behavior
\item Insider Threat: Ibis users should not be required to entrust us with their
  money. As a result, we have taken steps to ensure that it impossible (assuming
  a majority of active and rational stake holders) for the Ibis organization
  itself to steal money.
\end{itemize}

The detailed implementaiton of these mitgation techniques are embedded in the
module description sections comprising the remainder of this document.

\section{Ibis}

The Ibis contract contains the main business logic to run the system. It reads
and writes critical user data from the standalone Core contract and inherits
ownership and votable function modifiers from the Restricted and Democratic
modules, respectively. The most commonly used functions in this module will be
the deposit, transfer, and withdraw procedures.

The contract also holds upgrade logic. When a contract is upgraded, a
pre-declared \emph{init} function of the new contract is invoked. The current
contract self-destructs, sending the balance of ether to the new
implementation. Except for special circumstances, upgrades are invoked by
multi-owner approval and must be approved by a simple majority stakeholder vote.

Finally, Ibis implements a somewhat involved mechanism for token freezing and
redistribution. The most pressing economic threat to Ibis is the ability to
trade Ibis tokens for other currencies via an authorized exchange as this would
in a sense invalidate the purpose of Ibis as a charity token. While this threat
cannot be entirely eliminated, this contract freezing mechanism allows the Ibis
organization to have some power to mitigate abuse.

An Ibis organization owner address may instantly freeze any account at
will. After a set amount of time has passed, frozen funds may be submitted for
distribution amoung charities with majority stakeholder approval. The funds are
then paritioned into more manageable sizes and distributed to random charity
addresses.

The account freezing feature is primarily a regulatory tool. However, it also
provides a convenient mechanism for fund recovery should an Ibis user lose his
or her private account key. As a result, we can guarantee in the strictess sense
that every piece of money that ever enters the Ibis ecosystem will indeed make
its way to a charity.

\section{Core}

The Core module defines a stand-alone smart contract to house the critical
application data needed to run the platform. This includes available user
balances, the list of approved charities, and frozen user balances. The Core
contract recognizes a single controller address at any time, which is the
address of the contract that enforces the Ibis business logic. The controller
address is able to edit a list of approved addresses that can modify the Core
data. In the first iteration of Ibis, there are no extra approved addresses
other than the main controller.

The data housed in the Core contract is intended to be persistent for the full
lifetime of the Ibis ecosystem.

\section{Restricted}

Logic for regulating authoritative ownership of the contracts is housed in the
Restricted module. The Restricted module recognizes a dynamic, homogenous set of
contract owners as well as a single persistent \emph{master address}.

Permissions for contract owners are fairly straight forward. More common
operations such as adding and removing charity addresses may be done by a single
owner. More sensitive operations such as changing system parameters or adding
and removing owners must be approved by a certain threshold of the owner
set. The majority of owner addresses will be kept in cold storage to minimize
the risk of compromise.

In the event that a malicious adversary gains access to owner keys, the worst
damage can potentially be done right away. As a result, we also implement a
\emph{delay} functionality whereby sensitive operations can be forced to wait
a certain number of blocks before they are executed. A majority owner vote can
overrule such a delayed operation.

Finally, we also account for the worst-case scenario where all owner keys are
somehow compromised. In the event that this happens, the master address, with
the approval of the voting mechanism described later, has the ability to trigger
a system \emph{nuke}. In the event that this happens, all notions of ownership
and charity statuses are revoked. Users of all types will be able to cash out
their Ibis tokens for Ether.

The combination of these defense-in-depth mitigation measures ensures that we
can recover from small degrees of corruption. Even in the event of total
compromise, users will never have their tokens stolen or locked up forever.

\section{Democratic}

The measures implemented in the Restricted contract are sufficiently robust
against external hacking. However, we would like to ensure that user do not have
to trust the Ibis organization either.

Without voting abilities, the Ibis organization can trivially trigger a protocol
upgrade that sends all ether to a new address of our choosing. To mitigate this
abuse, the most signifcant operations (such as contract upgrades) must be
subject to user voting.

To vote on an ongoing issue, users must temporarily lock up their tokens. In
return, they will receive a proportional number of votes which they can apply
toward supporting or rejecting any active votable issues. Users may withdraw
their vote to retreive Ibis tokens at any time. When the voting period is
closed, only the remaining active votes are counted to see whether a proposed
operation should be executed.

Note that even frozen accounts are entitled to votes. This is a necessary
measure to prevent the Ibis organization from sensoring dissenting users.

The democratic paradigm should instill a level of confidence in the fact that
the Ibis organization cannot reasonably steal any user tokens. At the same time,
votable issues can only be triggered by the contract owners, ensuring that the
system cannot be corrupted by a tyranny of the majority. The evolution of Ibis
will trend toward the status quo and any changes can only be realized through a
robust system of checks and balances.

\section{Conclusion}

Security and trustworthess in the domain of charity is perhaps more important
than any other. By carefully considering all attack vectors, we believe we have
successfully designed an optimal balance in protecting the Ibis ecosystem. As
always, if you have any questions, concerns, or suggestions, please feel free to
raise them on any of our available forums. Thanks for reading!

\end{document}
